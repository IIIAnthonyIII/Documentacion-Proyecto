\documentclass[10pt,a4paper]{article}
\usepackage[utf8]{inputenc}
\usepackage{amsmath}
\usepackage{amsfonts}
\usepackage{amssymb}
\begin{document}
\begin{center}
\textbf{Universidad de Guayaquil}

\textbf{Facultad de Ciencias Matematicas y Fisicas}

Proyecto de Implantación de un Sistema de Gestión de la Calidad ISO 9001:2015 en la Empresa SCHNELL SOFTWARE S.L.

\textbf{Integrantes:}

Del Valle Anthony

Lozano Peter

Curso: SOF-S-MA-3-2
\end{center}

\textbf{Empresa: }SCHNELL SOFTWARE S.L.

\textbf{¿Qué es ISO?}

La Organización Internacional para la Estandarización conocida como ISO por sus siglas en inglés “International Standarization Organization”, creada en 1947, con sede en Ginebra (Suiza), tiene como principal objetivo promover la estandarización internacional para facilitar el intercambio de bienes y servicios, así como su desarrollo científico y tecnológico (Mora et al, 2012).

\textbf{¿Qué es la norma ISO 9001? Características y principios}

La ISO 9001 es una norma internacional que se aplica a los sistemas de gestión de calidad centrada en todos los elementos de administración de calidad con los que una empresa debe contar para tener un sistema efectivo que le permita administrar y mejorar la calidad de sus productos o servicios. Los clientes se inclinan por aquellas empresas que cuentan con esta acreditación porque de este modo se aseguran que la empresa seleccionada dispone de un buen sistema de gestión de calidad (Yáñez,2008).
Prosigue Yáñez (2008) que lo que caracteriza a la norma ISO 9001 es:
\begin{itemize}
\item Su enfoque basado en los procesos.
\item Su compatibilidad con otras normas de gestión.
\item Su gran énfasis en el cumplimiento de los requisitos legales y reglamentarios.
\item Su gran énfasis en la participación y compromiso de la alta dirección con la calidad.
\item El establecimiento de objetivos medibles en todas las funciones y niveles de la organización.
\item El seguimiento y análisis de la información que concierne a la satisfacción del cliente.
\item Mejora continua y análisis permanente de la eficacia del sistema de gestión de calidad.
\end{itemize}

El esquema renovado de la norma ISO 9001:2015 está dividido en 10 cláusulas que cubren todo el sistema de gestión.

El anexo SL es una abreviación de HIGH STRUCTURE LEVEL (HSL):

Cláusula 1 - Alcance

Cláusula 2 - Referencias

Cláusula 3 - Términos y definiciones

Cláusula 4 - Contexto de la organización

Cláusula 5 - Liderazgo

Cláusula 6 - Planificación

Cláusula 7 - Apoyo

Cláusula 8 - Operación (la única especificación)

Cláusula 9 - Evaluación de desempeño

Cláusula 10 - Mejoramiento

\begin{enumerate}
\item \textbf{Alcance}

La Empresa SCHNELL SOFTWARE S.L. comienza su actividad empresarial en el año 2001 con el objetivo de ofrecer un software especializado para las empresas de corte y doblado de acero, sector en crecimiento y constante evolución tecnológica. Desde entonces ha centrado su actividad en el desarrollo de programas que permitan optimizar el proceso de elaboración del hierro para hormigón armado.

La Empresa SCHNELL SOFTWARE S.L., trabaja en las siguientes áreas:
\begin{itemize}
\item Ingeniería
\item Producción
\item Materias primas
\item Logística
\item Gestión
\item Integración
\end{itemize}

Esta empresa está ubicada en Calle Fray Luis Amigo, 4 - PRINCIPAL OFICINA A, Zaragoza, 50006 , Zaragoza

\textbf{Teléfono:} +34 976 30 19 17

\textbf{Provincia: }Zaragoza (España)

\textbf{Email: }admon@schnellsoftware.net

\textbf{Sitio web: }https://www.schnellsoftware.com/

\item \textbf{Normas de referencia}

El Sistema de Gestión de la Calidad implantado en la Empresa SCHNELL SOFTWARE S.L., está basado en las siguientes normas:
\begin{itemize}
\item UNE-EN ISO 9000:2015 Sistema de gestión de la calidad. Fundamentos y vocabulario.
\item UNE-EN ISO 9001:2015 Sistema de gestión de la calidad. Requisitos.
\item UNE-EN ISO 19011:2012 Directrices para la auditoría de los sistemas de gestión.
\end{itemize}

\item \textbf{Términos y definiciones}

\textbf{ERP: }Sistema de planificación de recursos empresariales.

\textbf{PYMES: }Pequeña y mediana empresa.



Debilidades
Falta de áreas de desarrollo y comerciales.
No cuenta con plan de respaldo
No existe algún plan de capacitación para estudiante o pasantes.

Fortalezas
Está presente en las principales ferias de maquinaria y aplicaciones de software para ferralla
Cuenta con un grupo profesional de 15 personas ilusionadas en su trabajo, además de colaboradores tecnológicos y personas altamente cualificadas en sistemas de cálculo.
Actualmente Schnell Software está presente en todos los continentes, con más de 850 instalaciones y más de 4.500 licencias de software.

Amenazas
Problemas con respecto a los requerimientos de usuario
Costos de desarrollos 
Implementación de programas en diferentes países.

Oportunidades
Ofrecen un software adaptado a las necesidades de los clientes
Mantienen una filosofía de creatividad e innovación 
Revisan periódicamente este sistema de gestión con el fin de garantizar su eficacia y mejora continua.

Misión
Integrar soluciones digitales para la competitividad de las organizaciones en alianza con empresas complementarias con base en la vivencia de valores.

Visión
Ser un referente en el medio de las TIC´S, manteniendo siempre la vanguardia e incursionando en nuevas tecnologías.
 
Valores
Honestidad, transparencia, comunicación, trabajo en equipo, integridad y Responsabilidad Social.

Sistema de gestión de la calidad

Sistema de gestión de la calidad
Requisitos generales
Requisitos de la documentación
Generalidades
Manual de calidad
Control de los documentos
Control de los registros

Preguntas
¿La empresa de servicios ha establecido, documentado e implementado un Sistema de gestión de la calidad? 
¿Se tienen identificados los procesos y sus interacciones?
¿Se utilizan criterio y métodos que garanticen que los procesos y su control sean eficaces?
¿Disponen de los recursos necesarios, así como de información que se utilice para apoyar a la operación y el seguimiento de los procesos?
¿Se implantan las acciones necesarias para lograr resultados planificados y la mejora continua de los procesos? 
¿Se cuenta con algún documento en que se exprese la política y los objetivos de la calidad?
¿Posee un manual de calidad en el que se referencien los procesos y procedimientos, así como el alcance del sistema de gestión de la calidad?
¿La empresa posee los procedimientos documentados requeridos para el sistema de gestión de la calidad?
¿Se establece un procedimiento documentado para definir los controles necesarios para la disposición de los registros y documentos? 
\end{enumerate}
\end{document}
