\documentclass[10pt,a4paper]{article}
\usepackage[utf8]{inputenc}
\usepackage{amsmath}
\usepackage{amsfonts}
\usepackage{amssymb}
\begin{document}
\begin{center}
\textbf{Universidad de Guayaquil}

\textbf{Facultad de Ciencias Matematicas y Fisicas}

Proyecto de Implantación de un Sistema de Gestión de la Calidad ISO 9001:2015 en la Empresa SCHNELL SOFTWARE S.L.

\textbf{Integrantes:}

Del Valle Anthony

Lozano Peter

Curso: SOF-S-MA-3-2
\end{center}

\textbf{Empresa: }SCHNELL SOFTWARE S.L.

La Empresa SCHNELL SOFTWARE S.L. comienza su actividad empresarial en el año 2001 con el objetivo de ofrecer un software especializado para las empresas de corte y doblado de acero, sector en crecimiento y constante evolución tecnológica. Desde entonces ha centrado su actividad en el desarrollo de programas que permitan optimizar el proceso de elaboración del hierro para hormigón armado.

La Empresa SCHNELL SOFTWARE S.L., trabaja en las siguientes áreas:
\begin{itemize}
\item Ingeniería
\item Producción
\item Materias primas
\item Logística
\item Gestión
\item Integración
\end{itemize}

Esta empresa está ubicada en Calle Fray Luis Amigo, 4 - PRINCIPAL OFICINA A, Zaragoza, 50006 , Zaragoza

\textbf{Teléfono:} +34 976 30 19 17

\textbf{Provincia: }Zaragoza (España)

\textbf{Email: }admon@schnellsoftware.net

Responsabilidades


Debilidades
Falta de áreas de desarrollo y comerciales.
No cuenta con plan de respaldo
No existe algún plan de capacitación para estudiante o pasantes.

Fortalezas
Está presente en las principales ferias de maquinaria y aplicaciones de software para ferralla
Cuenta con un grupo profesional de 15 personas ilusionadas en su trabajo, además de colaboradores tecnológicos y personas altamente cualificadas en sistemas de cálculo.
Actualmente Schnell Software está presente en todos los continentes, con más de 850 instalaciones y más de 4.500 licencias de software.

Amenazas
Problemas con respecto a los requerimientos de usuario
Costos de desarrollos 
Implementación de programas en diferentes países.

Oportunidades
Ofrecen un software adaptado a las necesidades de los clientes
Mantienen una filosofía de creatividad e innovación 
Revisan periódicamente este sistema de gestión con el fin de garantizar su eficacia y mejora continua.

Misión
Integrar soluciones digitales para la competitividad de las organizaciones en alianza con empresas complementarias con base en la vivencia de valores.

Visión
Ser un referente en el medio de las TIC´S, manteniendo siempre la vanguardia e incursionando en nuevas tecnologías. 
Valores
Honestidad, transparencia, comunicación, trabajo en equipo, integridad y Responsabilidad Social.

\end{document}
