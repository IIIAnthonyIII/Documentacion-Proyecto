\documentclass[10pt,a4paper]{article}
\usepackage[utf8]{inputenc}
\usepackage{amsmath}
\usepackage{amsfonts}
\usepackage{amssymb}
\begin{document}
\begin{center}
\textbf{Universidad de Guayaquil}

\textbf{Facultad de Ciencias Matematicas y Fisicas}

Proyecto de Implantación de un Sistema de Gestión de la Calidad ISO 9001:2015 en la Empresa SCHNELL SOFTWARE S.L.

\textbf{Integrantes:}

Del Valle Anthony

Lozano Peter

Curso: SOF-S-MA-3-2
\end{center}

\textbf{Empresa: }SCHNELL SOFTWARE S.L.

\textbf{¿Qué es ISO?}

La Organización Internacional para la Estandarización conocida como ISO por sus siglas en inglés “International Standarization Organization”, creada en 1947, con sede en Ginebra (Suiza), tiene como principal objetivo promover la estandarización internacional para facilitar el intercambio de bienes y servicios, así como su desarrollo científico y tecnológico (Mora et al, 2012).

\textbf{¿Qué es la norma ISO 9001? Características y principios}

La ISO 9001 es una norma internacional que se aplica a los sistemas de gestión de calidad centrada en todos los elementos de administración de calidad con los que una empresa debe contar para tener un sistema efectivo que le permita administrar y mejorar la calidad de sus productos o servicios. Los clientes se inclinan por aquellas empresas que cuentan con esta acreditación porque de este modo se aseguran que la empresa seleccionada dispone de un buen sistema de gestión de calidad (Yáñez,2008).
Prosigue Yáñez (2008) que lo que caracteriza a la norma ISO 9001 es:
\begin{itemize}
\item Su enfoque basado en los procesos.
\item Su compatibilidad con otras normas de gestión.
\item Su gran énfasis en el cumplimiento de los requisitos legales y reglamentarios.
\item Su gran énfasis en la participación y compromiso de la alta dirección con la calidad.
\item El establecimiento de objetivos medibles en todas las funciones y niveles de la organización.
\item El seguimiento y análisis de la información que concierne a la satisfacción del cliente.
\item Mejora continua y análisis permanente de la eficacia del sistema de gestión de calidad.
\end{itemize}

El esquema renovado de la norma ISO 9001:2015 está dividido en 10 cláusulas que cubren todo el sistema de gestión.

El anexo SL es una abreviación de HIGH STRUCTURE LEVEL (HSL):

Cláusula 1 - Alcance

Cláusula 2 - Referencias

Cláusula 3 - Términos y definiciones

Cláusula 4 - Contexto de la organización

Cláusula 5 - Liderazgo

Cláusula 6 - Planificación

Cláusula 7 - Apoyo

Cláusula 8 - Operación (la única especificación)

Cláusula 9 - Evaluación de desempeño

Cláusula 10 - Mejoramiento

\begin{enumerate}
\item \textbf{Alcance}

La Empresa SCHNELL SOFTWARE S.L. comienza su actividad empresarial en el año 2001 con el objetivo de ofrecer un software especializado para las empresas de corte y doblado de acero, sector en crecimiento y constante evolución tecnológica. Desde entonces ha centrado su actividad en el desarrollo de programas que permitan optimizar el proceso de elaboración del hierro para hormigón armado.

La Empresa SCHNELL SOFTWARE S.L., trabaja en las siguientes áreas:
\begin{itemize}
\item Ingeniería
\item Producción
\item Materias primas
\item Logística
\item Gestión
\item Integración
\end{itemize}

Esta empresa está ubicada en Calle Fray Luis Amigo, 4 - PRINCIPAL OFICINA A, Zaragoza, 50006 , Zaragoza

\textbf{Teléfono:} +34 976 30 19 17

\textbf{Provincia: }Zaragoza (España)

\textbf{Email: }admon@schnellsoftware.net

\textbf{Sitio web: }https://www.schnellsoftware.com/

\item \textbf{Normas de referencia}

El Sistema de Gestión de la Calidad implantado en la Empresa SCHNELL SOFTWARE S.L., está basado en las siguientes normas:
\begin{itemize}
\item UNE-EN ISO 9000:2015 Sistema de gestión de la calidad. Fundamentos y vocabulario.
\item UNE-EN ISO 9001:2015 Sistema de gestión de la calidad. Requisitos.
\item UNE-EN ISO 19011:2012 Directrices para la auditoría de los sistemas de gestión.
\end{itemize}

\item \textbf{Términos y definiciones}

\textbf{ERP: }Sistema de planificación de recursos empresariales.

\textbf{PYMES: }Pequeña y mediana empresa.

\item \textbf{Contexto de la organización}

\textbf{Misión}

Integrar soluciones digitales para la competitividad de las organizaciones en alianza con empresas complementarias con base en la vivencia de valores.

\textbf{Visión}

Ser un referente en el medio de las TIC´S, manteniendo siempre la vanguardia e incursionando en nuevas tecnologías.Desarrollar acciones que favorezcan la especialización, las posibilidades de inserción laboral.

La Empresa SCHNELL SOFTWARE S.L. realiza el seguimiento y la revisión de su Misión y Visión, a través de las reuniones de
Dirección, basándose en los cambios que se producen en su entorno y los resultados obtenidos de sus procesos, así como la evolución de su Plan Estratégico, empleando para ello diferentes herramientas como análisis DAFO, estudios de mercado, análisis de la competencia, etc.

\textbf{Valores}

Honestidad, transparencia, comunicación, trabajo en equipo, integridad y Responsabilidad Social.

\textbf{Identificación del contexto y los grupos de interés.}

\textbf{Grupo de interés}

La Empresa SCHNELL SOFTWARE S.L. distingue dos tipos de grupos de interés:

Los internos que incluyen los accionistas, directivos, desarrolladores y trabajadores.

Los externos en donde se encuentran los clientes, proveedores, entidades financieras, comunidad local, organizaciones, etc.

Para un correcto análisis se implementó un esquema DAFO (Debilidades, Amenazas, Fortalezas, Oportunidades) que se muestra a continuación.

\textbf{Debilidades}

\begin{itemize}
\item Falta de áreas de desarrollo y comerciales.
\item No cuenta con plan de respaldo.
\item No existe algún plan de capacitación para estudiante o pasantes.
\end{itemize}

\textbf{Fortalezas}

\begin{itemize}
\item Está presente en las principales ferias de maquinaria y aplicaciones de software para ferralla.
\item Cuenta con un grupo profesional de 15 personas ilusionadas en su trabajo, además de colaboradores tecnológicos y personas altamente cualificadas en sistemas de cálculo.
\item Actualmente Schnell Software está presente en todos los continentes, con más de 850 instalaciones y más de 4.500 licencias de software.
\end{itemize}

\textbf{Amenazas}

\begin{itemize}
\item Problemas con respecto a los requerimientos de usuario.
\item Costos de desarrollos.
\item Implementación de programas en diferentes países.
\end{itemize}

\textbf{Oportunidades}

\begin{itemize}
\item Ofrecen un software adaptado a las necesidades de los clientes.
\item Mantienen una filosofía de creatividad e innovación. 
\item Revisan periódicamente este sistema de gestión con el fin de garantizar su eficacia y mejora continua.
\end{itemize}


Sistema de gestión de la calidad

Sistema de gestión de la calidad
Requisitos generales
Requisitos de la documentación
Generalidades
Manual de calidad
Control de los documentos
Control de los registros

Preguntas
¿La empresa de servicios ha establecido, documentado e implementado un Sistema de gestión de la calidad? 
¿Se tienen identificados los procesos y sus interacciones?
¿Se utilizan criterio y métodos que garanticen que los procesos y su control sean eficaces?
¿Disponen de los recursos necesarios, así como de información que se utilice para apoyar a la operación y el seguimiento de los procesos?
¿Se implantan las acciones necesarias para lograr resultados planificados y la mejora continua de los procesos? 
¿Se cuenta con algún documento en que se exprese la política y los objetivos de la calidad?
¿Posee un manual de calidad en el que se referencien los procesos y procedimientos, así como el alcance del sistema de gestión de la calidad?
¿La empresa posee los procedimientos documentados requeridos para el sistema de gestión de la calidad?
¿Se establece un procedimiento documentado para definir los controles necesarios para la disposición de los registros y documentos? 
Gestión de los recursos

Gestión de los recursos 
Provisión de recursos
Recursos humanos
Generalidades
Competencia, toma de conciencia y formación
Infraestructura
Ambiente del trabajo

Preguntas
¿Se determinan y proporcionan los recursos necesarios para mantener el sistema y mejorar su eficacia?
¿El personal de la organización tiene las competencias necesarias para la prestación del servicio?
¿Tiene el personal directivo las competencias necesarias para liderar?
¿Se mantiene el día los registros de formación, habilidades, experiencias y competencias de los empleados?
¿Están los trabajadores motivados y satisfechos con las funciones asignadas?
¿Es la infraestructura de la organización adecuada para asegurar el logro de la satisfacción del cliente?
¿cuenta la organización con el espacio de trabajo, los equipos y servicios de apoyo necesarios para la prestación del servicio?
¿Se determina y se gestiona el ambiente de trabajo necesario para lograr la conformidad con la prestación del servicio?

Aviso Legal
En cumplimiento de la Ley 34/2002 de 11 de julio, de Servicios de la Sociedad de la Información y del Comercio Electrónico le comunicamos que el titular de este sitio web es:
SCHNELL SOFTWARE S.L.
C.I.F. B-50879725
C/ Fray Luis Amigó 4, Pral. Of. A, “Edif. Rubí”
50006 ZARAGOZA
Tfno.: 976 301 917 Fax: 976 233 336
Inscrita en el Registro Mercantil de Zaragoza, Tomo 2669, Folio 64, Sección 8, Inscripción 2ª
email: admon@schnellsoftware.net
web: https://www.schnellsoftware.com
Schnell Software S.L. pone este sitio web a disposición de los usuarios de Internet, para proporcionar información sobre sus productos y/o servicios, ofrecer demostraciones de productos así como permitir a los usuarios realizar cualquier tipo de consulta o aportación a través de correo electrónico o los formularios incluidos.
La navegación por este sitio web lleva consigo la aceptación de todas las consideraciones expuestas en este documento.
Los criterios seguidos a efectos de este documento por Schnell Software S.L. respecto a la utilización de los datos personales facilitados libre y voluntariamente por los usuarios, son los que se mencionan en su política de privacidad.
Schnell Software S.L. podrá efectuar, en cualquier momento y sin necesidad de previo aviso, modificaciones y actualizaciones sobre la información contenida en este sitio web o en su configuración o presentación. En consecuencia, no garantiza la plena eficacia de su página web ni la inexistencia de algunos errores. Schnell Software S.L. se compromete a modificar su sitio web cuando las circunstancias técnicas y organizativas lo estimen oportuno y en la mayor brevedad posible.
Del mismo modo Schnell Software S.L. no se responsabiliza de los daños directos o indirectos que puedan derivarse del uso de la página web o sus contenidos. Tampoco se hace responsable de los daños informáticos inclusive virus que pudieran ocasionar al usuario visitante el acceso a los contenidos de este sitio.
El usuario se compromete a utilizar el sitio web y sus contenidos correctamente. Queda prohibido el uso del sitio web con fines ilícitos o lesivos contra la empresa o cualquier tercero, o aquellos que puedan causar perjuicio o impedir el normal funcionamiento del sitio web. Si contraviene la legislación vigente, la buena fe, usos, costumbres o el orden público el usuario será el responsable.
Este sitio web es propiedad de Schnell Software S.L. Los derechos de propiedad intelectual e industrial y derechos de explotación y reproducción de este sitio web, de sus páginas, pantallas, la información que contienen, su apariencia y diseño son propiedad exclusiva de la empresa salvo que se especifique otra cosa. Todas las denominaciones, diseños, fotografías y/o logotipos que componen esta página son marcas debidamente registradas. Cualquier uso indebido de las mismas por persona diferente de su legítimo titular podrá ser perseguido de conformidad con la legislación vigente. Los derechos de propiedad intelectual e industrial y marcas de terceros están destacados convenientemente y deben ser respetados por todo aquel que acceda al sitio web. Solo para uso personal y privado o de evaluación de productos se permite descargar los contenidos, copiar o imprimir cualquier página de este sitio web. Queda prohibido reproducir, transmitir, modificar o suprimir la información, contenido u advertencias de este sitio web sin la previa autorización escrita de Schnell Software S.L.
Todo enlace de terceros a este sitio web debe hacerse conforme a la buena fe y respetando los derechos del titular, debiendo comunicarse previamente a Schnell Software S.L. El enlace deberá realizarse a la página principal del sitio web, evitando las técnicas de “framing”.
Cualquier uso de un vínculo o acceso a un sitio web que no sea propiedad de Schnell Software S.L. es realizado por voluntad y riesgo exclusivo del usuario y Schnell Software S.L. no es responsable de ninguna información obtenida por o a través de un vínculo o mal funcionamiento de éste al acceder a la información de otros sitios web desde este sitio web.
En caso de que quiera formular alguna sugerencia o comentario puede contactar con nosotros como ya se ha mencionado a través de un correo electrónico a info@schnellsoftware.net
En caso de controversia respecto de cualquier materia del presente aviso legal, éste será sometido a la jurisdicción de los Tribunales y Juzgados de Zaragoza (España) siempre que ello no contravenga la legislación vigente. El presente aviso legal se regirá e interpretará de acuerdo con la legislación y jurisdicción española.

Política de privacidad
Schnell Software S.L. se compromete a proteger la privacidad de sus datos personales de acuerdo con la normativa vigente en materia de Protección de Datos (Ley Orgánica 15/1999, de 13 de Diciembre, de Protección de Datos de Carácter Personal, en adelante L.O.P.D.)
Schnell Software S.L. mantiene los niveles de seguridad de protección de sus datos conforme al Real Decreto 994/1999, de 11 de Junio, relativo a las medidas de seguridad de los ficheros automatizados que contengan datos de carácter personal. Estos niveles serán los adecuados a los datos que se traten. En ningún caso los datos serán cedidos a otras empresas ni estarán disponibles para terceros.
Envío de correo electrónico: Schnell Software S.L. pone a disposición del usuario las direcciones de correo electrónico mostradas en el apartado “contacto” para que realicen las consultas que estimen convenientes. Los datos personales introducidos serán incorporados al fichero “Usuarios página web” cuyo responsable es Schnell Software S.L. con la finalidad de resolver su petición y poder mantenerle informado de nuestros servicios o productos en un futuro a no ser que Ud. alegue lo contrario. En el caso de no suministrar todos los datos estimados necesarios esta entidad podrá no proceder al registro del usuario o denegar el servicio solicitado.
El usuario tiene derecho de acceso, rectificación, cancelación y oposición escribiendo a la dirección de correo electrónico info@schnellsoftware.net o a la dirección postal indicada en el aviso legal.
Envío de currículum: Usted puede hacer llegarnos su currículum a través del correo electrónico info@schnellsoftware.net indicado en la sección “Trabaja con nosotros”. Los datos proporcionados por los usuarios  en sus currículum deberán ser exactos, actuales y veraces. Serán procesados por Schnell Software S.L.  incorporados al fichero “Curricula” cuyo responsable es Schnell Software S.L., con la finalidad de contar con su historial profesional para desarrollar las funciones de selección y contratación.
El usuario tiene derecho de acceso, rectificación, cancelación y oposición dirigiendo una comunicación escrita en los términos indicados en la normativa sobre protección de datos a la dirección de correo electrónico info@schnellsoftware.net o a la dirección postal indicada en el aviso legal. Si pasado un año desde la inclusión del currículum no ha tenido noticias nuestras, procederemos al borrado de los datos de nuestro fichero.
El envío de un correo electrónico de consulta supone el consentimiento expreso del remitente para el tratamiento de los datos incluidos en los citados ficheros, así como para el envío de correos electrónicos a la dirección indicada.

Política de Calidad
SCHNELL SOFTWARE ofrece sus productos software al sector del acero y es consciente de la importancia de prestar un servicio tanto en el desarrollo como en la instalación y post-instalación del mismo, bajo unos criterios de calidad elevados.
Por ello, la Dirección de SCHNELL SOFTWARE apuesta por el desarrollo e implantación de un sistema de gestión de la calidad adecuado a la naturaleza de sus actividades y establece los siguientes principios de actuación:
Ofrecer un software adaptado a las necesidades de los clientes.
Mantener una filosofía de creatividad e innovación y trabajar en la mejora continua, para adaptarse a un sector en constante evolución tecnológica.
Trabajar con un sistema de gestión de calidad que establece nuestra forma de actuar en cuanto a la calidad de nuestro servicio, fijando objetivos y metas que ayuden a mejorar, y dotando de los medios humanos y económicos a nuestro alcance para conseguirlos.
Revisar periódicamente este sistema de gestión con el fin de garantizar su eficacia y mejora continua. 
Cumplir con todos los requisitos legales que nos sean de aplicación, cualquier otro requisito aplicable y aquellos que suscribamos voluntariamente.
Proporcionar un entorno estimulante y agradable que facilite un trabajo de calidad de nuestro personal en SCHNELL SOFTWARE, con un espíritu de trabajo en equipo y de servicio al cliente.
Fomentar la competencia y formación de nuestro equipo técnico.
Mantener una buena relación con nuestros proveedores, como colaboradores de importancia para nuestra empresa.
La Dirección de SCHNELL SOFTWARE comunica y difunde la presente política por todo nuestro equipo, por todas las personas que trabajan en su nombre y se asegura de que se encuentre a disposición de clientes y público en general y de que sea revisada para asegurar su vigencia y adecuación.

Política de Cookies
1. ¿Qué son las cookies?
Una cookie es un fichero que se descarga en su ordenador al acceder a determinadas páginas web. Las cookies permiten a una página web, entre otras cosas, almacenar y recuperar información sobre los hábitos de navegación de un usuario o de su equipo y, dependiendo de la información que contengan y de la forma en que utilice su equipo, pueden utilizarse para reconocer al usuario.
2. ¿Qué tipos de cookies utiliza esta página web?
La aplicación que utilizamos para obtener y analizar la información de la navegación es: Google Analytics: www.google.com/analytics/ y https://www.google.com/analytics/learn/privacy.html?hl=es
Esta aplicación ha sido desarrollada por Google, que nos presta el servicio de análisis de la audiencia de nuestra página. Esta empresa puede utilizar estos datos para mejorar sus propios servicios y para ofrecer servicios a otras empresas. Puedes conocer esos otros usos desde los enlaces indicados.
Cookies de análisis: Son aquéllas que bien tratadas por nosotros o por terceros, nos permiten cuantificar el número de usuarios y así realizar la medición y análisis estadístico de la utilización que hacen los usuarios del servicio ofertado. Para ello se analiza su navegación en nuestra página web con el fin de mejorar la oferta de productos o servicios que le ofrecemos.
Esta herramienta no obtiene datos de los nombres o apellidos de los usuarios ni de la dirección postal desde donde se conectan. La información que obtiene está relacionada por ejemplo con el número de páginas visitas, el idioma, red social en la que se publican nuestras noticias, la ciudad a la que está asignada la dirección IP desde la que acceden los usuarios, el número de usuarios que nos visitan, la frecuencia y reincidencia de las visitas, el tiempo de visita, el navegador que usan, el operador o tipo de terminal desde el que se realiza la visita.
Esta información la utilizamos para mejorar nuestra página, detectar nuevas necesidades y valorar las mejoras a introducir con la finalidad de prestar un mejor servicio a los usuarios que nos visitan.
3. ¿Cómo deshabilitar las Cookies?
Para permitir, conocer, bloquear o eliminar las cookies instaladas en tu equipo puedes hacerlo mediante la configuración de las opciones del navegador instalado en su ordenador.
Internet Explorer: Herramientas -> Opciones de Internet -> Privacidad -> Configuración.
Para más información, puede consultar el soporte de Microsoft o la Ayuda del navegador.
Firefox: Herramientas -> Opciones -> Privacidad -> Historial -> Configuración Personalizada.
Para más información, puede consultar el soporte de Mozilla o la Ayuda del navegador.
Chrome: Configuración -> Mostrar opciones avanzadas -> Privacidad -> Configuración de contenido.
Para más información, puede consultar el soporte de Google o la Ayuda del navegador.
Safari: Preferencias -> Seguridad.
Para más información, puede consultar el soporte de Apple o la Ayuda del navegador.
4. ¿Qué ocurre si se deshabilitan las Cookies?
Algunas funcionalidades de los Servicios quedarán deshabilitados como, por ejemplo, permanecer identificado, recibir información dirigida a su localización o la visualización de algunos vídeos.

Identificación del contexto y los grupos de interés
Schnell Software S.L. comienza su actividad con el objetivo de ofrecer un software especializado para las empresas. Una de las primeras tareas era llevar a cabo una planificación y posterior la implementación de un sistema de gestión de calidad acorde a ISO 9001.
Análisis del contexto
Contexto externo: Deberemos de tener en cuenta el entorno en el que nos movemos, así como las fuerzas competitivas que actúan en este sector “Informático, sistemático y digital”.
Centrándonos en el entorno encontramos varias variables que podrían afectarnos.
Variables macroeconómicas: nos permiten conocer la situación de la economía a nivel nacional lo que nos puede ayudar a sacar conclusiones útiles para el desarrollo de la empresa. La Oferta y la Demanda por encima de cualquier otra tienen gran incidencia porque basados en estos se estima los precios y posibilidades que se podrían ofrecer a los clientes.
Variables tecnológicas: nos movemos en un entorno tecnológico altamente desarrollado, en el que los usuarios demandan productos y servicios con un alto componente tecnológico. Contamos con la última tecnología en máquinas para nuestros clientes. 
Variables político- legales: En lo que se refiere a la legislación y reglamentación, la normativa genérica a aplicar es la siguiente: 
Prevención de Riesgos Laborales. Ley 31/1995, de 8 de noviembre y su modificación por la Ley 54/2003 de 12 de diciembre, de reforma del marco normativo de la prevención de Riesgos Laborales. BOE nº 298 de 13 de diciembre.
-Real Decreto 39/1997 por el que se establece el Reglamento de los Servicios de Prevención y Orden de 27 de junio de 1997 donde se desarrolla. 
- Real Decreto 485/97, de 14 de abril, en el que se indican las disposiciones mínimas en materia de señalización para la seguridad y salud en el trabajo.
- Real Decreto 486/97 sobre disposiciones mínimas de seguridad y salud en los lugares de trabajo.
- Real Decreto 488/97, de 14 de abril, sobre disposiciones mínimas de seguridad y salud en el trabajo que incluye pantallas de visualización. 
- Real Decreto 773/97 sobre equipos, sobre equipos de protección individual y demás disposiciones legales que afecten a la actividad. Además, serán de aplicación las siguientes normas específicas: 
- Decreto 2413/1973 del 20 de septiembre, por el que se aprueba el Reglamento electrónico de baja tensión. 
- Real decreto 2177/1996 del 4 de octubre, por el que se aprueba la Norma Básica de Edificación “NBE – CPI/96: Condiciones de Protección contra Incendios de los Edificios”. 
- Real Decreto 1316/1989 del 27 de octubre, sobre la protección de los trabajadores frente a los riesgos derivados de la exposición al ruido durante el trabajo.
En cuanto a las fuerzas competitivas que podrían afectarnos habrá que tener en cuenta: LA COMPETENCIA, entre centros de este tipo es cada vez mayor. Esto hace que cada vez resulta más complicado retener a un cliente debido a la creciente oferta de software.

Contexto interno: Nuestra organización la componen trabajadores altamente cualificados y con una dilatada experiencia en software y desarrollo de estos.
En cuanto a la Misión general de la empresa:
Schnell Software desarrolla un producto CAD-CAM que da solución a la organización y producción de industrias de ferralla, tanto de pequeño volumen (PYMES), como grandes industrias del acero a nivel internacional. Actualmente estamos presentes en todos los continentes, con más de 850 instalaciones y más de 4.500 licencias de software.
Respecto a la Visión general de la empresa:
Schnell Software S.L. comienza su actividad empresarial en el año 2001 con el objetivo de ofrecer un software especializado para las empresas de corte y doblado de acero, sector en crecimiento y constante evolución tecnológica. Desde entonces ha centrado su actividad en el desarrollo de programas que permitan optimizar el proceso de elaboración del hierro para hormigón armado.

\end{enumerate}
\end{document}
